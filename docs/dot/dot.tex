\documentclass[a4paper, 11pt]{article}

\author{Sofiane MAMI}

\usepackage[french]{babel}
\usepackage[utf8]{inputenc}
\usepackage[T1] {fontenc}
\usepackage[backend=biber, maxbibnames=5, style=numeric, sorting=none]{biblatex}
\usepackage{csquotes}
\DeclareCiteCommand{\supercite}[\mkbibsuperscript]
  {\iffieldundef{prenote}
     {}
     {\BibliographyWarning{Ignoring prenote argument}}%
   \iffieldundef{postnote}
     {}
     {\BibliographyWarning{Ignoring postnote argument}}}
  {\usebibmacro{citeindex}%
   \textbf{\bibopenbracket\usebibmacro{cite}\bibclosebracket}}
  {\supercitedelim}
  {}

\addbibresource{refs.bib}

\let\cite=\supercite

\title{La compression de donn\'ees appliqu\'ee aux images}

\begin{document}
    
\begin{center}
    {\textbf {\LARGE DOT - La compression de donn\'ees appliqu\'ee aux images}} \\
    \vspace{3mm}
    {\small Sofiane MAMI}
\end{center}

\vspace{5mm}

\begin{itemize}
    \item $\bigl[$Novembre 2022 - Identification des différents procédés de compression, et découverte de l'algorithme JPEG, regroupant de nombreuses méthodes efficaces.$\bigr]$
    \item $\bigl[$Janvier 2023 - Renseignement sur des méthodes plus récentes, mais le manque d'implémentation/applications n'a pas vraiment permis de nous aventurer dans cette direction$\bigr]$
    \item $\bigl[$Février-Mars 2023 - Implémentation des différents procédés de compression et des outils nécessaire à l'encodage final$\bigr]$
    \item $\bigl[$Avril 2023 - Difficultés rencontrées vis-à-vis des standards du format JPEG, afin de pouvoir avoir des données fiables et représentatives pour les comparaisons.$\bigr]$
    \item $\bigl[$Mai 2023 - La lecture des normes JPEG \cite{jpegStd} a permis d'éclaircir les zones d'ombres du format JPEG et un encodeur et décodeur complets ont pu être réalisés. Les données ont ensuite pu être exploitées.$\bigr]$
\end{itemize}

\printbibliography[title=Références bibliographiques]

\end{document}